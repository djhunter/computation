% Options for packages loaded elsewhere
\PassOptionsToPackage{unicode}{hyperref}
\PassOptionsToPackage{hyphens}{url}
%
\documentclass[
  twoside]{article}
\usepackage{amsmath,amssymb}
\usepackage{iftex}
\ifPDFTeX
  \usepackage[T1]{fontenc}
  \usepackage[utf8]{inputenc}
  \usepackage{textcomp} % provide euro and other symbols
\else % if luatex or xetex
  \usepackage{unicode-math} % this also loads fontspec
  \defaultfontfeatures{Scale=MatchLowercase}
  \defaultfontfeatures[\rmfamily]{Ligatures=TeX,Scale=1}
\fi
\usepackage{lmodern}
\ifPDFTeX\else
  % xetex/luatex font selection
\fi
% Use upquote if available, for straight quotes in verbatim environments
\IfFileExists{upquote.sty}{\usepackage{upquote}}{}
\IfFileExists{microtype.sty}{% use microtype if available
  \usepackage[]{microtype}
  \UseMicrotypeSet[protrusion]{basicmath} % disable protrusion for tt fonts
}{}
\makeatletter
\@ifundefined{KOMAClassName}{% if non-KOMA class
  \IfFileExists{parskip.sty}{%
    \usepackage{parskip}
  }{% else
    \setlength{\parindent}{0pt}
    \setlength{\parskip}{6pt plus 2pt minus 1pt}}
}{% if KOMA class
  \KOMAoptions{parskip=half}}
\makeatother
\usepackage{xcolor}
\usepackage[left=1.0in, right=1.0in, top=0.8in, bottom=0.4in]{geometry}
\usepackage{graphicx}
\makeatletter
\def\maxwidth{\ifdim\Gin@nat@width>\linewidth\linewidth\else\Gin@nat@width\fi}
\def\maxheight{\ifdim\Gin@nat@height>\textheight\textheight\else\Gin@nat@height\fi}
\makeatother
% Scale images if necessary, so that they will not overflow the page
% margins by default, and it is still possible to overwrite the defaults
% using explicit options in \includegraphics[width, height, ...]{}
\setkeys{Gin}{width=\maxwidth,height=\maxheight,keepaspectratio}
% Set default figure placement to htbp
\makeatletter
\def\fps@figure{htbp}
\makeatother
\setlength{\emergencystretch}{3em} % prevent overfull lines
\providecommand{\tightlist}{%
  \setlength{\itemsep}{0pt}\setlength{\parskip}{0pt}}
\setcounter{secnumdepth}{-\maxdimen} % remove section numbering
\usepackage[charter]{mathdesign}
\usepackage{fancyhdr}
\pagestyle{fancy}
\lhead{CS/MA-135, Hunter, Spring 2024}
\rhead{Syllabus, Page \thepage}
\cfoot{}
\setlength{\headsep}{0.2in}

\usepackage{enumitem}

\setenumerate{leftmargin=*}

\newcommand{\vect}[1]{\mathbf{#1}} % vectors in bold face

%\newcommand{\vect}[1]{\mathbf{#1}\,} % vectors in bold face (need thin
                                     % space after)
%\newcommand{\vect}[1]{\vec{#1}} %vectors as arrows

\providecommand{\norm}[1]{\left\lvert#1\right\rvert} % norms as single lines
%\providecommand{\norm}[1]{\left\lVert#1\right\rVert} % norms as double lines

% bold or blackboard bold!?
\newcommand{\NN}{\mathbb{N}}
%\newcommand{\RR}{\mathbf{R}}
\newcommand{\RR}{\mathbb{R}}

% some useful abbreviations
\newcommand{\vx}{\vect{x}}
\newcommand{\vy}{\vect{y}}
\newcommand{\vz}{\vect{z}}
\newcommand{\vu}{\vect{u}}
\newcommand{\vv}{\vect{v}}
\newcommand{\vw}{\vect{w}}
\newcommand{\va}{\vect{a}}
\newcommand{\vb}{\vect{b}}
\newcommand{\vc}{\vect{c}}
\newcommand{\ve}{\vect{e}}
\newcommand{\vf}{\vect{f}}
\newcommand{\vF}{\vect{F}}
\newcommand{\vg}{\vect{g}}
\newcommand{\vh}{\vect{h}}
\newcommand{\vl}{\vect{l}}
\newcommand{\vm}{\vect{m}}
\newcommand{\vn}{\vect{n}}
\newcommand{\vp}{\vect{p}}
\newcommand{\vr}{\vect{r}}
\newcommand{\vs}{\vect{s}}
\newcommand{\vi}{\vect{i}}
\newcommand{\vj}{\vect{j}}
\newcommand{\vk}{\vect{k}}
\newcommand{\vzero}{\vect{0}}
% lower-case greek letters are handled differently: poor man's bold macro
%\newcommand{\vphi}{\pmb{\phi}}

\newcommand{\malename}{Westley} % recurring male name
\newcommand{\femalename}{Buttercup} % recurring female name
\usepackage{booktabs}
\usepackage{longtable}
\usepackage{array}
\usepackage{multirow}
\usepackage{wrapfig}
\usepackage{float}
\usepackage{colortbl}
\usepackage{pdflscape}
\usepackage{tabu}
\usepackage{threeparttable}
\usepackage{threeparttablex}
\usepackage[normalem]{ulem}
\usepackage{makecell}
\usepackage{xcolor}
\ifLuaTeX
  \usepackage{selnolig}  % disable illegal ligatures
\fi
\IfFileExists{bookmark.sty}{\usepackage{bookmark}}{\usepackage{hyperref}}
\IfFileExists{xurl.sty}{\usepackage{xurl}}{} % add URL line breaks if available
\urlstyle{same}
\hypersetup{
  hidelinks,
  pdfcreator={LaTeX via pandoc}}

\author{}
\date{\vspace{-2.5em}}

\begin{document}

\hypertarget{formal-languages-and-automata-csma-135-westmont-college-spring-2024}{%
\section{Formal Languages and Automata (CS/MA-135) Westmont College,
Spring
2024}\label{formal-languages-and-automata-csma-135-westmont-college-spring-2024}}

\hypertarget{what-is-this-course-about}{%
\subsection{What is this course
about?}\label{what-is-this-course-about}}

This course provides a mathematical introduction to the theory of
computation, including finite automata, regular languages, context-free
languages, the Church-Turing thesis, decidability, reducibility, and
complexity. These topics are mainly theoretical, but they do have some
important applications in compiler design, natural language processing,
and software development. They also raise some interesting philosophical
questions.

\hypertarget{can-you-explain-it-another-way}{%
\subsection{Can you explain it another
way?}\label{can-you-explain-it-another-way}}

OK, here's the catalog description: ``(Four credit hours) Prerequisite:
CS/MA-015 Discrete Mathematics. In his early thirties, Alan Turing
cracked the Enigma code, thereby shortening WWII by two to four years.
As a consequence of this work, he established the theoretical foundation
for what is computable, and what is not. At the age of 41, he tragically
died from eating a poisoned apple. This course explores what it means to
compute, what features are necessary for a machine to compute, and the
respective limits on what different machines can compute.''

\hypertarget{how-is-the-course-structured}{%
\subsection{How is the course
structured?}\label{how-is-the-course-structured}}

Interactive

You should expect a written \textbf{prework} assignment after every
class meeting. These assignments will be due before the beginning of the
next class meeting. Each problem on the prework will include a list of
students who are responsible for \textbf{presenting} the problem; one of
the students on this list will be chosen to explain the problem to the
class. I expect you to have mostly complete solutions to all the
questions on the prework, but your solutions do not have to be perfect
to receive full credit. However, you should be very confident that you
understand and have a correct solution to the problems you are assigned
to present, because you may be required to explain your reasoning and
answer questions. Please come to Student Hours if you need help with the
prework.

\hypertarget{is-there-a-textbook}{%
\subsection{Is there a textbook?}\label{is-there-a-textbook}}

We will be working through the excellent \emph{Introduction to the
Theory of Computation}, by Michael Sipser, 3nd Edition. (The 2nd Edition
of this text will also suffice.)

\hypertarget{how-are-grades-determined}{%
\subsection{How are grades
determined?}\label{how-are-grades-determined}}

\begin{tabular}[t]{ll}
\toprule
Written Prework Assignments & 20\%\\
Presentations and Participation & 15\%\\
Exams (3) & 15\% each\\
Final Exam & 20\%\\
\bottomrule
\end{tabular}

\hypertarget{what-other-policies-should-students-be-aware-of}{%
\subsection{What other policies should students be aware
of?}\label{what-other-policies-should-students-be-aware-of}}

If you miss a significant number of classes, you will almost definitely
do poorly in this class. If you miss more than six classes without a
valid excuse, I reserve the right to terminate you from the course with
a failing grade. Work missed (including tests) without a valid excuse
will receive a zero.

I expect you to check your email on a regular basis. If you use a
non-Westmont email account, please forward your Westmont email to your
preferred account. I'll send out notices on Canvas, so make sure you
receive Canvas notifications in your email.

Learning communities function best when students have academic
integrity. Cheating is primarily an offense against your classmates
because it undermines our learning community. Therefore, dishonesty of
any kind may result in loss of credit for the work involved and the
filing of a report with the Provost's Office. Major or repeated
infractions may result in dismissal from the course with a failing
grade. Be familiar with the College's plagiarism policy, found at
\url{https://www.westmont.edu/office-provost/academic-program/academic-integrity-policy}.

In particular, providing someone with an electronic copy of your work is
a breach of the academic integrity policy. Do not email, post online, or
otherwise disseminate any of the work that you do in this class. If you
keep your work on a repository, make sure it is private. You may work
with others on the assignments, but make sure that you write or type up
your own answers yourself. You are on your honor that the work you hand
in represents your own understanding.

\clearpage

\hypertarget{other-information}{%
\subsection{Other Information}\label{other-information}}

\begin{description} 

\item[Professor:] David J. Hunter, Ph.D.
  (\verb!dhunter@westmont.edu!). Student hours are on Tuesdays from 1:00--3:20pm and Thursdays from 2:00--4:40pm in Winter Hall 303.

 \item[Tentative Schedule:] The following schedule is a rough first approximation of the topics in \textit{Sipser} that we plan to cover; it is subject to revision at the instructor's discretion. Chapter 0 (Mathematical Notions and Terminology) should be familiar to you from Discrete Mathematics, but these topics will be reviewed as necessary when they arise.
  \begin{itemize}
      \item Chapter 1: Regular Languages
         \begin{quote}
          \textit{Hour Exam \#1}     (through Chapter 1)
         \end{quote}
      \item Chapter 2: Context-Free Languages
      \item Chapter 3: The Church-Turing Thesis
         \begin{quote}
          \textit{Hour Exam \#2}     (through Chapter 3)
         \end{quote}
      \item Chapter 4: Decidability
      \item Chapter 5: Reducibility
         \begin{quote}
          \textit{Hour Exam \#3}     (through Chapter 5)
         \end{quote}
      \item Chapter 7: Time Complexity 
   \begin{quote}
    \textit{Final Exam}     (cumulative)
   \end{quote}
  \end{itemize}

\clearpage

\item[Accommodations for Students with Disabilities:] Students who choose to disclose a disability, diagnosis, or injury are encouraged to contact the Office of Disability Services (ODS) as early as possible in the semester to discuss potential accommodations for this course. Formal accommodations will only be granted for students whose diagnoses have been verified by the ODS. Accommodations are designed to minimize the impact of a diagnosis and to ensure equal access to programs for all students who have a diagnosed condition that impacts their participation in college activities. Please contact \href{mailto:ods@westmont.edu}{\tt ods@westmont.edu} or visit the website for more information: \url{https://www.westmont.edu/disability-services-welcome}. Seth Miller, Director of ODS and the ODS team are located upstairs in Voskuyl Library 310, 311A. 

\item[Program and Institutional Learning Outcomes:] The
         mathematics department at Westmont College has formulated the
         following learning outcomes for all of its classes. (PLO's)
\begin{enumerate}[noitemsep]
\item Core Knowledge: Students will demonstrate knowledge of the
                  main concepts, skills, and facts of the discipline of
                  mathematics.
\item Communication: Students will be able to communicate mathematical ideas
     following the standard conventions of writing or speaking in the
     discipline.
\item Creativity: Students will demonstrate the ability to formulate and make
     progress toward solving non-routine problems.
\item Christian Connection: Students will incorporate their mathematical skills
     and knowledge into their thinking about their vocations as followers of
     Christ.
         \end{enumerate}
         In addition, the faculty of Westmont College have established common
         learning outcomes for all courses at the institution
         (ILO's). These outcomes are summarized as follows:
(1) Christian Understanding, Practices, and Affections,
(2) Global Awareness and Diversity,
(3) Critical Thinking,
(4) Quantitative Literacy,
(5) Written Communication,
(6) Oral Communication, and
(7) Information Literacy.

\begin{itemize}
    \item Demonstrate understanding of the theoretical basis for languages and computation.
             (PLO 1, ILOs 3,4)
    \item Write and evaluate mathematical arguments according to the
             standards of the discipline. (PLO 2,
              ILOs 3,5)
    \item Construct solutions to novel problems,
               demonstrating perseverance in the face of open-ended or
               partially-defined contexts. (PLO 3, ILO 3)
    \item Consider the theological implications of the subject matter. (PLO 4, ILO 1)
\end{itemize}
These outcomes will be assessed by written assignments, presentations, and written exams, as described above.

\end{description}

\end{document}
