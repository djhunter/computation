\documentclass[10pt]{article}

\pagestyle{myheadings}
\usepackage[dvips, letterpaper, total={6.5in,9.0in}]{geometry}
\parindent 0 in
\usepackage[charter]{mathdesign}
\usepackage{hyperref}

\markright{CS/MA-135, Hunter, Spring 2020}

\let\origdescription\description
\renewenvironment{description}{
  \setlength{\leftmargini}{0em}
  \origdescription
  \setlength{\itemindent}{0em}
}

\begin{document}
\begin{center} {\large CS/Math 135 --- Formal Languages and Automata}
\end{center}

\begin{description}
\item[Time and place:] 9:15--10:20am, MWF, WH 110
\item[Professor:] David J. Hunter, Ph.D.

\hspace*{20pt}
\begin{tabular}{ll}
  e-mail: & \href{mailto:dhunter@westmont.edu}{\tt dhunter@westmont.edu} \\
  Office: & Winter Hall 303 \\
  Student Hours: & MTWTh 2:00-4:00pm, or by appointment.
\end{tabular}

\item[Catalog Description:] (Four credit hours) Prerequisite: CS/MA-015 Discrete Mathematics.  In his early  thirties,  Alan Turing cracked the Enigma code, thereby shortening WWII by two to four years. As a consequence of this work, he established the theoretical foundation for what is computable, and what is not. At the age of 41, he tragically died from eating a poisoned apple. This course explores what it means to compute, what features are necessary for a machine to compute, and the respective limits on what different machines can compute.

\item[Overview:] This course provides a mathematical introduction to the theory of computation, including finite automata, regular languages, context-free languages, the Church-Turing thesis, decidability, reducibility, and complexity. These topics are mainly theoretical, but they do have some important applications in compiler design, natural language processing, and software development. They also raise some interesting philosophical questions.

\item[Textbook:] \textit{Introduction to the Theory of Computation}, by Michael Sipser, 3nd Edition. (The 2nd Edition of this text will also suffice.)

\item[Grading:] Grades are weighted as follows.

\begin{tabular}{ll}
  Written Prework: & 20\% \\
  Presentations: & 15\% \\
  Hour Exams: & 3 @ 15\% each \\
  Final Exam: & 20\%
\end{tabular}

\textbf{Exams} will be written, without the use of anything electronic. The final exam will be on Thursday, April 30, from 8:00--10:00am. Finals will not be rescheduled to accommodate travel arrangements.

You should expect a written \textbf{prework} assignment after every class meeting. These assignments will be due at the beginning of the next class meeting. Please complete the prework on the paper provided, adding additional sheets if necessary, so that they can be scanned when you arrive in class. Each problem on the prework will include a list of students who are responsible for \textbf{presenting} the problem; one of the students on this list will be chosen to explain the problem to the class. I expect you to have mostly complete solutions to all the questions on the prework, but your solutions do not have to be perfect to receive full credit. However, you should be very confident that you understand and have a correct solution to the problems you are assigned to present, because you may be required to explain your reasoning and answer questions. Please come to Student Hours if you need help with the prework.

\item[Attendance:]  Please show up to class on time; it is rude and distracting to your classmates if you come to class late.  Significant tardies count as absences.  If you miss an excessive number of classes, you will almost definitely do poorly in this class.  I consider it excessive to miss more than three classes during the course of the semester.  If you miss more than six classes without a valid excuse, I reserve the right to terminate you from the course with a grade of~F.\hspace{4pt}  Work missed (including tests) without a valid excuse will receive a zero.

\item[Other Policies: ]
  Learning communities function best when students have academic integrity.  Cheating is primarily an offense against your classmates because it undermines our learning community.  Therefore, dishonesty of any kind may result in loss of credit for the work involved and the filing of a report with the Provost's Office. Major or repeated infractions may result in dismissal from the course with a grade of~F.~ Be familiar with the College's plagiarism policy, found at: \url{https://www.westmont.edu/office-provost/academic-program/academic-integrity-policy}.

  In particular, providing someone, actively or passively, with a copy of your work is a breach of the academic integrity policy. Do not email, post online, or otherwise disseminate any of the work that you do in this class. You may work with others on the assignments, but make sure that you write up your own answers yourself. You are on your honor that the work you hand in represents your own understanding.

\clearpage

 \item[Tentative Schedule:] The following schedule is a rough first approximation of the topics in \textit{Sipser} that we plan to cover; it is subject to revision at the instructor's discretion. Chapter 0 (Mathematical Notions and Terminology) should be familiar to you from Discrete Mathematics, but these topics will be reviewed as necessary when they arise.
  \begin{itemize}
      \item Chapter 1: Regular Languages
         \begin{quote}
          \textit{Hour Exam \#1}     (through Chapter 1)
         \end{quote}
      \item Chapter 2: Context-Free Languages
      \item Chapter 3: The Church-Turing Thesis
         \begin{quote}
          \textit{Hour Exam \#2}     (through Chapter 3)
         \end{quote}
      \item Chapter 4: Decidability
      \item Chapter 5: Reducibility
         \begin{quote}
          \textit{Hour Exam \#3}     (through Chapter 5)
         \end{quote}
      \item Chapter 7: Time Complexity (if time permits)
   \begin{quote}
    \textit{Final Exam}     (cumulative)
   \end{quote}
  \end{itemize}

    \item[Program and Institutional Learning Outcomes:] The
         mathematics and computer science department at Westmont College has formulated the
         following learning outcomes for all of its classes. (PLO's)
         \begin{enumerate}
             \item Core Knowledge: Students will demonstrate knowledge of the
                  main concepts, skills, and facts of the discipline of
                  mathematics.
\item Communication: Students will be able to communicate mathematical ideas
     following the standard conventions of writing or speaking in the
     discipline.
\item Creativity: Students will demonstrate the ability to formulate and make
     progress toward solving non-routine problems.
\item Christian Connection: Students will incorporate their mathematical skills
     and knowledge into their thinking about their vocations as followers of
     Christ.
         \end{enumerate}
         In addition, the faculty of Westmont College have established common
         learning outcomes for all courses at the institution
         (ILO's). These outcomes are summarized as follows:
(1) Christian Understanding, Practices, and Affections,
(2) Global Awareness and Diversity,
(3) Critical Thinking,
(4) Quantitative Literacy,
(5) Written Communication,
(6) Oral Communication, and
(7) Information Literacy.

\item[Course Learning Outcomes:] The above outcomes are reflected in the
     particular learning outcomes for this course.
     After taking this course, you should be able
     to:
    \begin{itemize}
        \item Demonstrate understanding of the theoretical basis for languages and computation.
             (PLO 1, ILOs 3,4)
        \item Write and evaluate mathematical arguments according to the
             standards of the discipline. (PLO 2,
              ILOs 3,5)
        \item Construct solutions to novel problems,
               demonstrating perseverance in the face of open-ended or
               partially-defined contexts. (PLO 3, ILO 3)
        \item Consider the theological implications of the subject matter. (PLO 4, ILO 1)
    \end{itemize}
These outcomes will be assessed by written assignments, presentations, and written exams, as described above.

\item[Accommodations for Students with Disabilities:] Students who have been diagnosed with a disability (learning, physical or psychological) are strongly encouraged to contact the Disability Services office as early as possible to discuss appropriate accommodations for this course. Formal accommodations will only be granted for students whose disabilities have been verified by the Disability Services office.  These accommodations may be necessary to ensure your equal access to this course.  Please contact Sheri Noble, Director of Disability Services (310A Voskuyl Library, 565-6186, \href{mailto:snoble@westmont.edu}{\tt snoble@westmont.edu}) or visit \url{https://www.westmont.edu/disability-services} for more information.
\end{description}
\end{document}
